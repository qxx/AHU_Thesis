\def\Q{\noindent Question:~~}
\def\A{\newline Answer:~~}
\chapter{FAQ}
\label{chap:contact}

这里是我们已经收到的一些常见问题以及一些一般的Debug技巧. 对于更多的信息或者关于本模板和文档的任何疑问,请移步\href{http://bbs.ustc.edu.cn/cgi/bbsdoc?board=TeX}{瀚海星云\TeX{}讨论区}。

\Q 从哪里下载\CTeX{}套装? 应该下载哪一个\CTeX{}套装?
\A 在\href{http://www.ctex.org}{\CTeX{}官方网站}可以下载, 进入网站后点击网页右边的``项目''栏目中``最新稳定版'', 即可进入下载链接. 对于有良好网络链接的同学, 我们强烈建议下载较大的那个完全版, 实在是网络速度不佳的同学可以下载精简版的, 但是编译的时候会经常实时下载某些宏包, 很麻烦, 因此不推荐.

\Q 我自己新建了一个文件并且写了一段文字, 为什么编译出来的中文是乱码呢? 但是用模板里面的文件编译正常.
\A 请仔细确认你新建的文件是不是UTF-8编码! 如果不是, 请按文章前述方法转换成UTF-8编码.

\Q 我编译总是报错, 按e定位总是定位到主文件?
\A 这种问题比较复杂, 可能是你的某个正文中的命令与模板冲突, 或者有时与宏包冲突也会出现这种情况. 这时我们只能手动定位错误, 具体方式如下: 首先确定问题在哪一章? 方法是一章一章的编译, 然后确定问题在哪一节, 方法是只编译问题章节, 然后一节一节的编译. 最后确定问题在哪一行. 一般对于新手而言, 容易在数学公式中出问题, 所以, 请优先检查数学公式. 这样通过逐级定位, 我们就可以精确的获知错误的位置, 然后仔细检查, 就可以得到错误的原因, 即使你无法理解错误, 也可以对症求医, 在\TeX{}版求问.

\Q 编译错误时, DOS窗口里报了一大堆英文, 如何看懂啊?
\A 前面的都是编译信息不用理睬, 后面的是错误信息要仔细看, 比如这一段
\begin{code}
  ! Undefined control sequence.
l.16 \adga
           agasgarga
?
\end{code}
这一段有4行, 最后一行是问你如何处理, 我们在这里输入我们的处理方案. 现在看前三行. 第一行由一个感叹号开头, 这里是错误原因, 说明我用了一个没有定义的命令.
第二行是\verb|l.16|开头, 说明第16行出问题了, 然后这里有一个断行, 这个断行表示错误的精确位置, 这里, 由于没有\verb|\adga|这个命令, 自然就会报错.
后面的第三行是辅助文字, 表示错误后面的一些代码, 这样可以方便我们精确定位, 这种显示方式在大段的数学公式出错时尤其方便. 至于第4行, 前述办法是: 一般按E, 回车, 方便的在代码中回溯到我们需要的行. 在这里我们再介绍一个命令: h---回车. h的意思是help, 这样, \TeX{}的编译器会使用尽可能你熟悉的语言告诉你具体为什么错了. 有的时候这是很有帮助的, 而有的时候你却完全不知道他在说什么. 这是很正常的, 因为电脑的长处在于执行命令, 而不是主动读懂你的想法.(Knuth in TeXbook)

\Q 一个诡异的错误:
\A 也许你已经学习了一些公式的编写, 那么, 请看下面一段代码:
\begin{code}
\begin{eqnarray}
  [X_1, X_2] &=& 0\\
  [X_3, X_4] &=& Q_{\pm}
\end{eqnarray}
\end{code}
你肯定觉得这是个很简单的括号问题, 但是实际上这是无法编译的! 解决方案是在第一行公式后面加上\verb|\relax|, 原因很巧妙, 换行符\verb|\\|后面的中括号代表该命令的可选参数, 表示空多少尺寸, 在上述公式中, 第二行的公式恰好是以\verb|[|开头, 自然无法编译过去. 这在TeXbook中叫做``weird mistakes''. 其实, 我们有很多可能遇到各种诡异的问题, 有时这种问题的思路是超越我们的想法的, 因此, 希望大家可以多多发问, 善于在网络上与同学们讨论问题, 这样才能有助于问题的解决.


