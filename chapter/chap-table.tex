
\chapter{表格}
\label{chap:tab}
标准三线表:
\begin{table}[!ht]
\label{tab:table3}
\caption{三线表格示例}
\centering
\begin{tabular}{c c c c}
\whline 一列 & 两列 & 三列 & 四列  \\\hline
1 & 2 &3 &4 \\2 & 3 &4 &5 \\3 & 4 &5 &6 \\
\whline
\end{tabular}
\end{table}
\begin{code}
    \begin{table}[!ht]
    \label{tab:table3}
    \caption{三线表格示例}
    \centering
    \begin{tabular}{c c c c}
        \whline 一列 & 两列 & 三列 & 四列  \\\whline
        1 & 2 &3 &4 \\2 & 3 &4 &5 \\3 & 4 &5 &6 \\
        \whline
    \end{tabular}
    \end{table}
\end{code}

与figure一样,table环境也是浮动体,常用!ht来表明位置。caption、label的意义都与figure环境相同。

而画表格的tabular环境则与array环境几乎相同,只是在每行之间可以用\textbackslash{}hline和\textbackslash{}hline这些命令来画横线。c代表一个居中对齐的列,l代表靠左对齐,r代表靠右对齐。
有竖线的例子:
\begin{table}[!ht]
\label{tab:table2}
\caption{表格示例2}
\centering
\begin{tabular}{ Ic|c|c|cI}
\whline 一列 & 两列 & 三列 & 四列  \\\hline
1 & 2 &3 &4 \\\hline 2 & 3 &4 &5 \\\hline3 & 4 &5 &6 \\
\whline
\end{tabular}
\end{table}
\begin{code}
    \begin{table}[!ht]
    \label{tab:table2}
    \caption{表格示例2}
    \centering
    \begin{tabular}{ Ic|c|c|cI}
        \whline 一列 & 两列 & 三列 & 四列  \\\hline
        1 & 2 &3 &4 \\\hline 2 & 3 &4 &5 \\\hline3 & 4 &5 &6 \\
        \whline
    \end{tabular}
    \end{table}
\end{code}

即竖线是|,粗竖线是大写字母I,横线是\textbackslash hline,粗横线是\textbackslash whline。
