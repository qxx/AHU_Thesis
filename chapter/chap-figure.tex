
\chapter{图形}
\label{chap:fig}


\section{浮动图形}

\XeLaTeX 支持jpg和eps格式的图片。我们所用的编译方法支持jpg和eps等格式的图片。如果你已经有一篇word文档,
想把其中的图全部导出,那么可以将它另存为html文件,这时会在同目录下找到一个文件夹,里面是所有用到的图片。

将插图放在figures文件夹中,想要引用的地方使用类似这样的命令插入图形文件:
\begin{code}
\begin{figure}[!ht]
 \centering
 \includegraphics[width=0.2\textwidth]{ustc_logo_new.eps}
 \caption{中国科学技术大学校徽(在页面中间)}
 \label{fig:ustc1}
\end{figure}
\end{code}
结果如下:
\begin{figure}[!ht]
 \centering
 \includegraphics[width=0.2\textwidth]{ustc_logo_new.eps}
 \caption{中国科学技术大学校徽(在页面中间)}
 \label{fig:ustc1}
\end{figure}

在上面这段命令中,可选参数[!ht]代表插图的位置。其中!让\LaTeX{}忽略审美标准,试图用最严格的标准来放置浮动图形;h(ere)代表有限放在此处;t(op)代表如果此处放不下,那么放在下一页页首。

width=0.2\textbackslash{}textwidth代表图形的宽度是文字宽度的0.2倍,你也可以使用其他长度单位,如12cm,4.5in等等。

caption命令的参数代表图的名称,或者说注解。label的参数用于交叉引用,见下节。


\section{交叉引用}
通过使用交叉引用功能,我们可以定位那些\LaTeX{}自动编号的内容,比如浮动图形、表格、章节等等。其中,label和ref的参数是引用的名字,可以随意,但须保持一致。
\begin{figure}[ht]
\centering
\fbox{\begin{minipage}[h]{0.4\textwidth}
引用图\ref{fig:ustc1}\\
引用第\ref{chap:introduction}章
\end{minipage}}
\hspace{0.1\textwidth}
\begin{minipage}[h]{0.4\textwidth}
\centering
\begin{code}
引用图\ref{fig:ustc1}\\
引用第\ref{chap:introduction}章
\end{code}
\end{minipage}
\caption{交叉引用示例}
\end{figure}


\section{并列的子图}
使用subfigure命令,一个例子:
\begin{figure}[!hbt]
\centering
\subfigure[sf 1]{
\includegraphics[width=0.2\textwidth]{ustc_logo_new.eps}\label{f:1}}
\subfigure[sf 2]{
\includegraphics[width=0.2\textwidth]{ustc_logo_new.eps}\label{f:2}}
\subfigure[sf 3]{
\includegraphics[width=0.2\textwidth]{ustc_logo_new.eps}\label{f:3}}
\subfigure[sf 4]{
\includegraphics[width=0.2\textwidth]{ustc_logo_new.eps}\label{f:4}}
\caption{\label{f:s}subfigure使用示例。}
\end{figure}
\begin{code}
\begin{figure}[!hbt]
\centering
\subfigure[sf 1]{
\includegraphics[width=0.2\textwidth]{ustc_logo_new.eps}\label{f:1}}
\subfigure[sf 2]{
\includegraphics[width=0.2\textwidth]{ustc_logo_new.eps}\label{f:2}}
\subfigure[sf 3]{
\includegraphics[width=0.2\textwidth]{ustc_logo_new.eps}\label{f:3}}
\subfigure[sf 4]{
\includegraphics[width=0.2\textwidth]{ustc_logo_new.eps}\label{f:4}}
\caption{\label{f:s}subfigure使用示例。}
\end{figure}
\end{code}

分两行放:
\begin{figure}[!hbt]
\centering
\subfigure[sf 1]{
\includegraphics[width=0.2\textwidth]{ustc_logo_new.eps}\label{f:1}}
\subfigure[sf 2]{
\includegraphics[width=0.2\textwidth]{ustc_logo_new.eps}\label{f:2}}\\
\subfigure[sf 3]{
\includegraphics[width=0.2\textwidth]{ustc_logo_new.eps}\label{f:3}}
\subfigure[sf 4]{
\includegraphics[width=0.2\textwidth]{ustc_logo_new.eps}\label{f:4}}
\caption{\label{f:s}subfigure使用示例。}
\end{figure}
\begin{code}
\begin{figure}[!hbt]
\centering
\subfigure[f 1]{
\includegraphics[width=0.2\textwidth]{ustc_logo_new.eps}\label{f:1}}
\subfigure[f 2]{
\includegraphics[width=0.2\textwidth]{ustc_logo_new.eps}\label{f:2}}\\
\subfigure[f 3]{
\includegraphics[width=0.2\textwidth]{ustc_logo_new.eps}\label{f:3}}
\subfigure[f 4]{
\includegraphics[width=0.2\textwidth]{ustc_logo_new.eps}\label{f:4}}
\caption{\label{f:l}分行的字图使用示例。}
\end{figure}
\end{code}